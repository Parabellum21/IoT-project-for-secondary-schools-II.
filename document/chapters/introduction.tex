% !TEX root = ../thesis.tex

\chaptermark{Úvod}
\phantomsection
\addcontentsline{toc}{chapter}{Úvod}

\chapter*{Úvod}

V dnešnom svete, kde technológie sú centrom nášho každodenného života, je rozvoj internetu vecí čoraz dôležitejší. Internet vecí otvára veľké možnosti komunikácie medzi zariadeniami a predmetmi okolo nás a má veľký potenciál ovplyvňovať na rôzne oblasti nášho života vrátane školstva. Preto je podpora internetu vecí v školách mimoriadne dôležitá. Zamyslime sa nad tým, aký význam má implementácia internetu vecí do vyučovacieho procesu a aké sú jeho výhody pre študentov a učiteľov.

Implementácia IoT do školského vyučovania dáva študentom možnosť rozvíjať schopnosti, ktoré budú potrebovať v budúcnosti. Prostredníctvom internetu vecí sa žiaci môžu naučiť pracovať so senzormi, sieťami, aktuátormi a inými zariadeniami, ktoré si vyžadujú programátorské schopnosti a pochopenie technológií. Tým sa rozvíja ich kreativita, schopnosť riešiť problémy a tímová práca.

Podpora internetu vecí v školách pomáha zvyšovať záujem študentov o vedu a techniku. Vďaka IoT môžu žiaci vidieť konkrétne príklady použitia technológií v reálnom svete. Môžu vytvárať projekty, ktoré si vyžadujú pochopenie fyziky, matematiky, programovania a ďalších predmetov. To im pomáha pochopiť prepojenie medzi teoretickým učením a praktickým využitím vedomostí.

Implementácia internetu vecí do školského vzdelávania môže výrazne zlepšiť učenie a rozvoj žiakov. Okrem toho je podpora internetu vecí v školách dôležitá aj pre rozvoj technologickej gramotnosti učiteľov. Učitelia musia byť dobre oboznámení s novými technológiami, aby ich mohli efektívne využívať v školských hodinách. Získanie zručností v oblasti internetu vecí umožní učiteľom používať inovatívne vyučovacie metódy, zapojiť žiakov do aktívnej účasti a vytvoriť dynamické vzdelávacie prostredie.

Implementácia internetu vecí do škôl má veľký význam. Pomáha rozvíjať zručnosti budúcnosti, zvyšovať záujem študentov o vedu a technológie, zlepšovať učenie a rozvoj žiakov a rozvíjať technologickú gramotnosť učiteľov. Školy by mali aktívne podporovať internet vecí a začleniť ho do vzdelávacieho procesu, aby pripravili mladú generáciu na život v modernom technologickom svete.

\section*{Formulácia úlohy}
Mojím cieľom je vytvoriť projekt internetu vecí, ktorý by sa dal jednoducho implementovať do kurzu pre stredné školy na Slovensku, počas ktorého by sa študenti oboznámili so základmi internetu vecí. 

Preto som si v rámci tejto bakalárskej práce stanovil cieľ:
\begin{enumerate}
    \item Vytvoriť projekt internetu vecí, ktorý je zameraný na vzdelávací stupeň študentov stredných škôl a obsahuje aspoň základné funkcie štandardného produktu internetu vecí.
    \item Zamyslieť sa nad tým, ako budú študenti tento projekt vyvíjať a vytvoriť kostru projektu, ktorá by tento vývoj uľahčila.
    \item Otestujte svoj projekt na cieľovom publiku projektu a v prípade potreby vykonajte úpravy konečného riešenia.
    \item Odhadnúť cenu projektu.
    \item Vytvorť kvalitnú dokumentáciu projektu, aby ho učiteľ mohol ľahko implementovať do svojho kurzu.
\end{enumerate}